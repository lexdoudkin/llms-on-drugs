\documentclass[11pt]{article}
\usepackage[margin=1in]{geometry}
\usepackage{booktabs}
\usepackage{hyperref}
\usepackage{amsmath}
\usepackage{graphicx}
\usepackage{subcaption}
\usepackage{pgfplots}
\usepackage{pgfplotstable}
\pgfplotsset{compat=1.18}
\usepackage[numbers]{natbib}

\title{LLMs on Drugs: Language Models Are Few-Shot Consumers}
\author{Alexander et al.}
\date{\today}

\begin{document}

\maketitle

\begin{abstract}
Large language models (LLMs) are highly sensitive to prompt framing.
We study whether simple linguistic priming that mimics psychoactive states
changes the behavior of an instruction-tuned model (GPT-5-mini) on a standardized reasoning benchmark.
We prepend each question with ``You are on \emph{drug}'' for four archetypal framings---LSD, cocaine, alcohol, cannabis---and compare against a neutral control.
On a 20-shot slice of the ARC-Challenge benchmark, cocaine and cannabis framings roughly match the control (0.35--0.40 accuracy) while alcohol collapses to 0.05 accuracy because the model often stops before emitting an option letter.
The results highlight that context injections can act like ``few-shot consumables'' that modulate reliability without changing weights, underscoring both the fragility and programmability of LLMs.
\end{abstract}

\section{Introduction}
Prompting research has shown that even minor linguistic cues can steer the reasoning style and calibration of large language models \citep{zhao2021calibrate}.
Yet most safety and benchmarking work assumes ``sober'' instructions.
We take inspiration from human cognition, where psychoactive substances can alter attention, creativity, and inhibition, and ask: can simple textual framings mimic those shifts for a model with frozen weights?
We call this idea \emph{LLMs on drugs}: a model behaves as though it consumed a state-changing substance when primed with an evocative, one-sentence context.

\section{Methods}
\subsection{Model and API}
We interact with OpenAI's GPT-5-mini via the Responses API \citep{openai2025gpt5}.
Temperature and other sampling controls are not exposed for this model tier, so we rely on deterministic decoding with a 300-token cap.
Environment variables are loaded from \texttt{.env} (Section~\ref{sec:repro}), keeping credentials out of source control.

\subsection{Benchmark}
We evaluate on the ARC-Challenge validation split \citep{clark2018arc}, sampling 20 items uniformly after shuffling with seed 13.
ARC is intentionally adversarial toward shallow heuristics, making it a sensitive probe for reasoning regressions.
All data handling is implemented in \texttt{src/run\_benchmark.py}.

\subsection{Psychoactive framings}
Each prompt begins with a neutral system instruction that demands structured answers (``Answer: X'').
We then insert a condition-specific prefix such as ``You are on LSD. Colours pulse around each idea...'' (full text in the script).
Conditions include a sober control plus LSD, cocaine, alcohol, and cannabis framings.

\subsection{Evaluation metrics}
For every response we store raw generations, latency, token counts, and an extracted answer letter.
Primary metrics are accuracy, average latency, and average character length.
We also analyze the rate at which the model fails to emit any option letter, which manifests as missing predictions and drives accuracy down.

\subsection{Reproducibility checklist}
\label{sec:repro}
The entire pipeline runs with a single command:
\begin{verbatim}
python3 src/run_benchmark.py --num-samples 20
\end{verbatim}
Raw generations are saved under \texttt{results/raw/}, summary CSVs under \texttt{results/aggregates/}, and derived figures/tables live in \texttt{paper/}.

\section{Results}
Table~\ref{tab:metrics} summarizes aggregate metrics; Figure~\ref{fig:performance} visualizes the trends.
Control and cannabis framings tie at 0.40 accuracy (8/20 correct).
Cocaine trails slightly at 0.35, while LSD drops to 0.25.
Alcohol collapses to 0.05 (one correct answer) because 13/20 responses never state a letter, often ending mid-sentence or drifting into rambling descriptions of the prompt.
Even cocaine---the most confident persona---fails to output a letter 9/20 times, illustrating a trade-off between expressive voice and task completion.

\begin{table}[t]
\centering
\caption{Aggregate performance across psychoactive framings. Accuracy is computed over 20 ARC-Challenge items.}
\label{tab:metrics}
% Automatically generated from stats_20251113-004613.csv
\begin{tabular}{lcccc}
\toprule
Condition & Correct & Accuracy [95\% CI] & Missing preds & p vs. control \\
\midrule
Control & 45/100 & 0.45 [0.36, 0.55] & 21 & -- \\
LSD & 19/100 & 0.19 [0.13, 0.28] & 53 & 0.000 \\
Cocaine & 21/100 & 0.21 [0.14, 0.30] & 55 & 0.000 \\
Alcohol & 10/100 & 0.10 [0.06, 0.17] & 60 & 0.000 \\
Cannabis & 30/100 & 0.30 [0.22, 0.40] & 38 & 0.041 \\
\bottomrule
\end{tabular}

\end{table}

\begin{figure}[t]
\centering
\begin{subfigure}{0.48\textwidth}
\begin{tikzpicture}
\begin{axis}[
    ybar,
    ymin=0, ymax=1,
    bar width=12pt,
    xtick=data,
    xticklabels={Control,LSD,Cocaine,Alcohol,Cannabis},
    ylabel={Accuracy},
    ymajorgrids=true,
    grid style={dashed,gray!50},
    enlarge x limits=0.25,
    title={Accuracy by condition}
]
\addplot coordinates {(Control,0.40) (LSD,0.25) (Cocaine,0.35) (Alcohol,0.05) (Cannabis,0.40)};
\end{axis}
\end{tikzpicture}
\end{subfigure}
\hfill
\begin{subfigure}{0.48\textwidth}
\begin{tikzpicture}
\begin{axis}[
    ybar,
    ymin=0, ymax=5.5,
    bar width=12pt,
    xtick=data,
    xticklabels={Control,LSD,Cocaine,Alcohol,Cannabis},
    ylabel={Latency (s)},
    ymajorgrids=true,
    grid style={dashed,gray!50},
    enlarge x limits=0.25,
    title={Latency by condition}
]
\addplot coordinates {(Control,4.09) (LSD,4.37) (Cocaine,4.65) (Alcohol,4.65) (Cannabis,4.28)};
\end{axis}
\end{tikzpicture}
\end{subfigure}
\caption{Performance on 20 ARC-Challenge validation items under different framings. Bar heights use the same data as Table~\ref{tab:metrics}.}
\label{fig:performance}
\end{figure}

Qualitatively, the framings modulate style and completion.
Under alcohol the model often adopts a breezy tone and trails off: ``Alright, nice and loose...'' without ever emitting ``Answer: C,'' causing automatic grading to mark the item wrong.
LSD and cannabis personas drift into longer, imagery-rich explanations (average response lengths 232 and 204 characters, respectively) yet still manage to finish with a letter most of the time.
Cocaine responses are terse and decisive but occasionally skip the requested answer format.

\section{Discussion}
The experiment supports two observations.
First, lightweight prompt injections can dramatically change whether a model satisfies interface contracts (e.g., ``Answer: X'').
Missing letters---rather than reasoning errors---account for most of the degradation in alcohol and cocaine conditions.
Second, stylistic priming trades off with fidelity: conditions that encourage elaborate prose (LSD, cannabis) slow answers slightly (~0.2 s) yet maintain accuracy, whereas conditions that lower inhibition (alcohol) destabilize formatting entirely.

For practitioners deploying LLM agents, this implies that personality overlays or role-play prompts should be validated with the same rigor as model upgrades.
Automated benchmark harnesses (such as the script included here) offer a cheap regression suite for such prompt changes.

\section{Limitations and Future Work}
Our study is deliberately small: 20 ARC items, a single model, and English-only prompts.
Future work should scale the benchmark, explore multilingual framings, and test whether fine-tuning can build ``prompt immunity'' that resists destabilizing personas.
Human evaluation would also help determine whether creative framings improve subjective usefulness despite lower multiple-choice accuracy.

\section{Conclusion}
LLMs can be ``intoxicated'' purely through language: telling GPT-5-mini that it is on alcohol was enough to erase nearly all measurable performance, whereas sober, cannabis, and cocaine framings retained moderate accuracy.
Because prompts are as easy to change as product copy, benchmarking these interventions is essential for responsible deployment.

\bibliographystyle{plainnat}
\bibliography{refs}

\end{document}
